%!TEX root = main.tex
\usepackage{xcolor}

\newcommand{\vpt}{\vspace{.1in}}
\newcommand{\s}{\mathcal{S}}
\newcommand{\one}{\mathbf{1}}
\newcommand{\half}{\frac{1}{2}}
\newcommand{\nhalf}{\nicefrac{1}{2}}
\newcommand{\Tol}{\texttt{Tol}}
\newcommand{\calA}{\mathcal{A}}
\newcommand{\calX}{\mathcal{X}}
\newcommand{\vx}{\mathbf{x}}
\newcommand{\hvx}{\hat{\mathbf{x}}}
\newcommand{\1}{\mathbbm{1}}

\newcommand{\maxs}[1]{\noindent{\textcolor{blue}{\{{\bf MS:} \em #1\}}}}
\newif\ifpx
\DeclareOption{px}{\pxtrue}

\ProcessOptions

\ifpx
  \let\iint\relax
  \let\iiint\relax
  \let\iiint\relax
  \let\idotsint\relax
  \usepackage[varg]{pxfonts}
\fi
% \DeclareOption{tx}{
%   \let\iint\relax
%   \let\iiint\relax
%   \let\iiint\relax
%   \let\idotsint\relax
%   \usepackage[varg]{txfonts}
% }

% Math delimiters
\DeclarePairedDelimiter{\abs}{\lvert}{\rvert} %
\DeclarePairedDelimiter{\brk}{[}{]}
\DeclarePairedDelimiter{\crl}{\{}{\}}
\DeclarePairedDelimiter{\prn}{(}{)}
\DeclarePairedDelimiter{\nrm}{\|}{\|}
\DeclarePairedDelimiter{\tri}{\langle}{\rangle}
\DeclarePairedDelimiter{\dtri}{\llangle}{\rrangle}

\DeclarePairedDelimiter{\ceil}{\lceil}{\rceil}
\DeclarePairedDelimiter{\floor}{\lfloor}{\rfloor}

% \DeclareMathOperator{\E}{\mathbb{E}} %expecation
\let\Pr\undefined
\let\P\undefined
\DeclareMathOperator*{\En}{\mathbb{E}}
\DeclareMathOperator{\P}{P}
\DeclareMathOperator{\Pr}{\mathbb{P}}

% Arg<x>
\DeclareMathOperator*{\argmin}{arg\,min} % * Places subscript directly under operator
\DeclareMathOperator*{\argmax}{arg\,max}             
\DeclareMathOperator*{\arginf}{arg\,inf} 
\DeclareMathOperator*{\argsup}{arg\,sup} 

% Sets
 \newcommand{\R}{\mathbb{R}}
 \newcommand{\N}{\mathbb{N}}
% \newcommand{\Z}{\mathbb{Z}}
% \newcommand{\Q}{\mathbb{Q}}
% \newcommand{\J}{\mathbb{J}}
% \newcommand{\C}{\mathbb{C}}

\newcommand{\ind}{\mathbbm{1}}    %Indicator

% \newcommand{\V}{\mathcal{V}}
% \newcommand{\X}{\mathcal{X}}
% \newcommand{\Y}{\mathcal{Y}}
% \newcommand{\A}{\mathcal{A}}
% \newcommand{\F}{\mathcal{F}}
% \newcommand{\G}{\mathcal{G}}
% \newcommand{\D}{\mathcal{D}}

\newcommand{\eps}{\epsilon}
%\newcommand{\defeq}{\overset{def}{=}}
%\newcommand{\defeq}{\triangleq}
\newcommand{\defeq}{\coloneqq}

\newcommand{\xr}[1][n]{x_{1:#1}}
\newcommand{\yr}[1][n]{y_{1:#1}}
\newcommand{\zr}[1][n]{z_{1:#1}}

\newcommand{\bigO}{\mathcal{O}}



% misc stuff
\newcommand{\fb}{{\bf{}f}}

\newcommand{\mc}[1]{\mathcal{#1}}
\newcommand{\reg}{\mc{R}}
\newcommand{\breg}{D_{\reg}}

\newcommand{\apx}{\emph{ApxReg}}

%Thodoris' additions
\newcommand{\cost}{\ensuremath{\mathit{cost}}}
\newcommand{\opt}{\text{\textsc{Opt}} }
\newcommand{\obj}{\text{\textsc{Obj}} }
\newcommand{\val}{\text{\textsc{val}} }
\newcommand{\sol}{\text{\textsc{sol}} }

%{
 %\theoremstyle{plain}
 %     \newtheorem{asm}{Assumption}
%}
\newtheorem{nono-theorem}{Theorem}[]

\theoremstyle{plain
}
\setlength{\topsep}{.1in} 
\newtheorem{theorem}{Theorem}[section]
\newtheorem{claim}[theorem]{Claim}
\newtheorem{lemma}[theorem]{Lemma}
\newtheorem{corollary}[theorem]{Corollary}
\newtheorem{fact}[theorem]{Fact}
%\theoremstyle{remark}
%\newtheorem*{rem}{Remark}
\newtheorem*{note}{Note}

\newtheorem{prop}[theorem]{Proposition}
\theoremstyle{definition}
\newtheorem{definition}{Definition}[section]
\newtheorem{conjecture}{Conjecture}[section]
\newtheorem{example}{Example}[section]
\newtheorem{exc}{Exercise}[section]
\newtheorem{rem}{Remark}[section]
\newtheorem{obs}{Observation}[section]


\newenvironment{rtheorem}[3][]{

\bigskip

\noindent \ifthenelse{\equal{#1}{}}{\bf #2 #3}{\bf #2 #3 (#1)}
\begin{it}
}{\end{it}}


\newcommand{\poly}{\mathrm{poly}}
\newcommand{\regret}{\mathrm{regret}}
\newcommand{\starhull}{\mathsf{star}}
\newcommand{\cone}{\mathsf{cone}}
\newcommand{\Exp}{\mathbf{E}}



\newcommand{\nk}{N_k}
\newcommand{\Lfactor}{\mathbf{L}}



